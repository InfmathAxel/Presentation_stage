\documentclass[10pt]{beamer}
\usetheme{Warsaw}
\usepackage[T1]{fontenc}
\usepackage[utf8]{inputenc}
\usepackage{chronosys}
\usepackage{graphicx}
\usepackage{multicol} 


\begin{document}

\begin{frame}
\frametitle{KINETIC AND BUILDING LOD2}
\begin{figure}

    \begin{minipage}{0.33\textwidth}
        \centering
        \includegraphics[width=0.6\textwidth]{../image/logo_irma.png}
    \end{minipage}%
    \begin{minipage}{0.33\textwidth}
        \centering
        \includegraphics[width=0.6\textwidth]{../image/logo-numpex-web-2.png}
    \end{minipage}%
    \begin{minipage}{0.33\textwidth}
        \centering
        \includegraphics[width=0.6\textwidth]{../image/logo-ufr-mathinfo-unistra-1.jpg}
    \end{minipage}
    
    \vspace{0.5cm} 
    
   
    \begin{minipage}{0.33\textwidth}
        \centering
        \includegraphics[width=0.6\textwidth]{../image/logoCemosis-square-e1469167314407.png}
    \end{minipage}%
    \begin{minipage}{0.33\textwidth}
        \centering
        \includegraphics[width=0.6\textwidth]{../image/HiDALGO2-logo.jpg}
    \end{minipage}%
    \begin{minipage}{0.33\textwidth}
        \centering
        \includegraphics[width=0.6\textwidth]{../image/logo-inria-.jpg}
    \end{minipage}
    
\end{figure}

\vspace{0.5cm} 

Intern: Demuth Axel \\
Supervisor: Vincent Chabannes, Pierre Alliez, Florent Lafarge

\end{frame}

\begin{frame}
    \frametitle{Table of Contents}
    \begin{multicols}{2} % Diviser la page en deux colonnes
        \tableofcontents
    \end{multicols}
    \end{frame}


\section{Introduction}
\begin{frame}{Introduction}
\end{frame}

\subsection{Context}
\begin{frame}{Context}
\begin{figure}
    \begin{minipage}{0.45\textwidth}
    \includegraphics[width=0.9\textwidth]{../image/mesh_with_issue.png}
    \caption{Mesh with issues}
    
\end{minipage}
    \begin{minipage}{0.45\textwidth}
    \includegraphics[width=0.9\textwidth]{../image/mesh_fixed.png}
    \caption{Mesh without issues}
\end{minipage}
\end{figure}
\end{frame}

\subsection{Issue with Kinetic Algorithm}
\begin{frame}{Issue with orientation}
    \begin{figure}
        \centering
        \begin{minipage}[b]{0.3\textwidth}
          \centering
          \includegraphics[width=\textwidth]{../image/cube_3faces.png}
          \caption{Cube not oriented}
        \end{minipage}%
        \begin{minipage}[b]{0.3\textwidth}
          \centering
          \includegraphics[width=\textwidth]{../image/cube_5faces.png}
          \caption{Cube badly oriented}
        \end{minipage}%
        \begin{minipage}[b]{0.3\textwidth}
          \centering
          \includegraphics[width=\textwidth]{../image/bon_cube.png}
          \caption{Cube oriented by CGAL}
        \end{minipage}
      \end{figure}
\end{frame}

\begin{frame}{Self Intersection issue}
    \begin{figure}
        \centering
        \begin{minipage}[b]{0.23\textwidth}
          \centering
          \includegraphics[width=\textwidth]{../image/2cubes.png}
          \caption{Two cube self intercting}
        \end{minipage}\hfill
        \begin{minipage}[b]{0.23\textwidth}
          \centering
          \includegraphics[width=\textwidth]{../image/3cubes.png}
          \caption{Two cubes intersecting a third one}
        \end{minipage}\hfill
        \begin{minipage}[b]{0.23\textwidth}
          \centering
          \includegraphics[width=\textwidth]{../image/3cubes_tresproche.png}
          \caption{Three cubes self intersecting}
        \end{minipage}\hfill
        \begin{minipage}[b]{0.23\textwidth}
          \centering
          \includegraphics[width=\textwidth]{../image/5cubes.png}
          \caption{Five cubes  intercting randomly}
        \end{minipage}
      \end{figure}    
\end{frame}

\begin{frame}
\begin{figure}
    \centering
    \includegraphics[width=0.4\textwidth]{../image/2_cubes_fixed.png}
    \caption{Two Cubes fixed}
  \end{figure}
All other result in a execution error
\end{frame}

\begin{frame}{Objectives}
    \subsection{Objectives}
    \begin{itemize}
        \item Check the validity of the Mesh
        \item Create a workflow for automatic generation using KSR Algorithm
        \item Keep the correspondence of surfaces between both meshes
        \item Run some simulations using the Feel++ library
    \end{itemize}
\end{frame}


\begin{frame}{CGAL}
    \subsection{CGAL}
    \begin{itemize}
        \item C++ library for geometric calculations, providing data structures for mesh generation and manipulation.
    \end{itemize}
    \vspace{0.5cm}
    The main packages utilized are:
    \begin{itemize}
        \item \texttt{CGAL::Polygon\_mesh\_processing}
        \item \texttt{CGAL::Surface\_mesh}
        \item \texttt{CGAL::Point\_set\_processing}
        \item \texttt{CGAL::IO\_streams}
        \item \texttt{CGAL::AABB\_tree}
    \end{itemize}
    
\end{frame}


\section{Data}
\subsection{Files Format}
\begin{frame}{File Format}
\begin{itemize}
    \item IFC : Standart for buillding data modeling,similar to class oriented code
    \item CityGML : 3D format for city modeling with representation of geographic details
    \item STL : 3D Modeling format 
    \item OBJ :A standard file format for 3D models
    \item OFF :  A file format for 3D mesh data
    \item PLY :  A file format for 3D mesh data,stocking the cloud point of the mesh 
    \item MSH : A file format for mesh data use by GMSH software
\end{itemize}
\end{frame}

\subsection{Software and Data}
\begin{frame}{Software}
    \begin{itemize}
        \item Github : Platforme for collaborating work on a project
        \item Visual Studio Code : Versatil tools for coding with various extensions
        \item Paraview : Open-source data analysis and visualisation
        \item Meshlab : A tool for processing,editing,visualisation of 3D mesh
        \item GMSH : a 3D finite element mesh generator
    \end{itemize}

\end{frame}

\begin{frame}{Data}
    The following Data were given by Vincent Chabannes
    \begin{figure}
        \begin{minipage}{0.33\textwidth}
            \centering
            \includegraphics[width=0.9\textwidth]{../image/3zones_stl.png}
        \end{minipage}%
        \begin{minipage}{0.33\textwidth}
            \centering
            \includegraphics[width=0.9\textwidth]{../image/label_three_zone.png}
        \end{minipage}
        \caption{Three zones mesh}
    \end{figure}
    \begin{figure}
        \begin{minipage}{0.33\textwidth}
            \centering
            \includegraphics[width=0.9\textwidth]{../image/jasmin_stl.png}
        \end{minipage}
        \begin{minipage}{0.33\textwidth}
            \centering
            \includegraphics[width=0.9\textwidth]{../image/ACJasmin.png}
        \end{minipage}
        \caption{ACJasmin mesh}
    \end{figure}
\end{frame}


\section{Methodology}
\subsection{Kinetic}
\begin{frame}{Kinetic}
We get information from a INRIA report (citer le rapport)
Kinetic algorithm is an geometric algorithm generate 3D mesh from a point clouds,it uses  geometric primitive with an energy based model to fit the primitives to the model.

Energy formule: 
\newline
\begin{center}
    $        U(x) = w_f U_f(x) + w_s U_s(x) + w_c U_c(x)       $
\end{center}

to calculate the best primitive to fit the mesh.
then we have a list of geometric operation on each primitive
\includegraphics[scale=0.35]{../image/primitives_operation.png}
\end{frame}

\subsection{preprocessing}
\begin{frame}{preprocessing}  
    To improve Kinetic outcome we pre-process the mesh :
    \begin{itemize}
        \item Isotropic remeshing of the mesh
        \item Unified and regularize the mesh with grid simplify
        \item Fix self Intersection 
        \item Calcul normals
    \end{itemize}
\end{frame}

\subsection{Labelling}
\begin{frame}{Labelling}
    \textbf{Issue}: Inria developed a method to preserve the semantic information of IFC elements, but it has not yet been implemented in CGAL.

     Two potential solutions:
    \begin{itemize}
        \item Modify the Kinetic Solver to recognize and utilize markers on each point used to form a shape.
        \item Compare the input and output meshes to apply the same markers to the closest faces.
    \end{itemize}
\end{frame}

\begin{frame}{Labelling}
    Exemple of result of second solutions:
    \begin{figure}
        \centering
        \begin{minipage}{0.5\textwidth}
            \centering
            \includegraphics[width=0.9\textwidth]{../image/ACJasmin.png}
            \caption{Input Mesh}
            \label{fig:input_mesh}
        \end{minipage}%
        \begin{minipage}{0.5\textwidth}
            \centering
            \includegraphics[width=1.05\textwidth]{../image/ACjasmin_label.png}
            \caption{Output Mesh}
            \label{fig:output_mesh}
        \end{minipage}
    \end{figure}
\end{frame}

\subsection{Metric}
\begin{frame}{Metric}
We also want to add method to check the quality off the output mesh
\begin{itemize}
    \item Properties Check (closed,connected,triangulated...)
    \item Correspondance between input and output
\end{itemize}
To check the Correspondance between mesh, we can compare 
bounding box of each labelled elements.

\begin{table}[h]
    \centering
    \caption{Bounding Box value}
    \begin{tabular}{|c|c|c|}
        \hline
         \% of marker correct & Three Zones & ACJasmin \\
        \hline
        <5\% & 22/57 & 3/82 \\
        between 5 and 10 \% & 11/57 & 7/82 \\
        between 10 and 20 \% & 13/57  & 9/82 \\
        \hline
    \end{tabular}
\end{table}
\end{frame}

\begin{frame}
    
    \begin{figure}
    \includegraphics[width=0.9\textwidth]{../image/3_zones_input_bounding_box.png}
    \includegraphics[width=0.9\textwidth]{../image/3zone_output_bounding_box.png}
    \caption{Three zones Bounding Boxe comparaison}
    \end{figure}
    
\end{frame}

\begin{frame}
    
    \begin{figure}
    \includegraphics[width=0.4\textwidth]{../image/ACJasmin_input_bounding_box.png}
    \includegraphics[width=0.47\textwidth]{../image/ACjasmin_output_bounding_box.png}
    \caption{ACJasmin Bounding Boxe comparaison}
    \end{figure}
    
\end{frame}

\section{Implementation}
\subsection{Contribution to Ktirio library}
\begin{frame}{Function implemented}
   \begin{itemize}
    \item checkProperties{}
    \item gridSimplify{}
    \item remesh{}
    \item  KSR{}
   \end{itemize}

\end{frame}

\subsection{test}
\begin{frame}{test}
    \begin{itemize}
        \item test on Surface Mesh Check
        \item test on Kinetic algorithm
        \item test on Point set class and manipulation function
    \end{itemize}
\end{frame}

\section{Result}
\subsection{Point cloud generation Result}
\begin{frame}{Point cloud}
    \begin{figure}
        \begin{minipage}{0.45\textwidth}
            \centering
            \includegraphics[width=0.9\textwidth]{../image/3zones_stl.png}
        \end{minipage}%
        \begin{minipage}{0.45\textwidth}
            \centering
            \includegraphics[width=0.9\textwidth]{../image/3zones_ply.png}
        \end{minipage}
        \caption{Three zones mesh point cloud}
    \end{figure}
\end{frame}

\begin{frame}
    \begin{figure}
        \begin{minipage}{0.45\textwidth}
            \centering
            \includegraphics[width=0.9\textwidth]{../image/jasmin_stl.png}
        \end{minipage}
        \begin{minipage}{0.45\textwidth}
            \centering
            \includegraphics[width=0.9\textwidth]{../image/jasmin_ply.png}
        \end{minipage}
        \caption{ACJasmin mesh point cloud }
    \end{figure}
\end{frame}

\begin{frame}{Comparison of Kinetic Outcome}
    \begin{figure}
      \centering
      \begin{minipage}[b]{0.48\textwidth}
        \centering
        \includegraphics[width=\textwidth]{../image/3zones_v2_1.png}
        \caption{Old KSR outcome}
      \end{minipage}
      \hfill
      \begin{minipage}[b]{0.48\textwidth}
        \centering
        \includegraphics[width=\textwidth]{../image/3zones_final.png}
        \caption{New KSR outcome}
      \end{minipage}
    \end{figure}
\end{frame}

\begin{frame}{Comparison of 3 Zone Results}
\begin{figure}
  \centering
  \begin{minipage}[b]{0.48\textwidth}
    \centering
    \includegraphics[width=\textwidth]{../image/3zone_500_0.06.png}
    \caption{3 Zone 500}
  \end{minipage}\hfill
  \begin{minipage}[b]{0.48\textwidth}
    \centering
    \includegraphics[width=\textwidth]{../image/3zones_45_0.06.png}
    \caption{3 Zones 45}
  \end{minipage}
\end{figure}
\end{frame}


\begin{frame}{Comparison of Jasmin Images}
    \begin{figure}
      \centering
      \begin{minipage}[b]{0.48\textwidth}
        \centering
        \includegraphics[width=\textwidth]{../image/jasmin_ply.png}
        \caption{Jasmin Ply}
      \end{minipage}\hfill
      \begin{minipage}[b]{0.48\textwidth}
        \centering
        \includegraphics[width=\textwidth]{../image/jasmin.png}
        \caption{Jasmin Image}
      \end{minipage}
    \end{figure}
    \end{frame}

\subsection{Self Intersection Result}
\begin{frame}{Self Intersection fixing}
    \begin{figure}
        \centering
        \begin{minipage}[b]{0.23\textwidth}
          \centering
          \includegraphics[width=\textwidth]{../image/2cubesksr.png}
          \caption{Same result as intro}
        \end{minipage}\hfill
        \begin{minipage}[b]{0.23\textwidth}
          \centering
          \includegraphics[width=\textwidth]{../image/3cubesksr2.png}
          \caption{Worked}
        \end{minipage}\hfill
        \begin{minipage}[b]{0.23\textwidth}
          \centering
          \includegraphics[width=\textwidth]{../image/3cubesksr.png}
          \caption{Worked}
        \end{minipage}\hfill
        \begin{minipage}[b]{0.23\textwidth}
          \centering
          \includegraphics[width=\textwidth]{../image/5cubeoutcome.png}
          \caption{Worked}
        \end{minipage}
      \end{figure}
\end{frame}

\begin{frame}
    \begin{figure}
      \centering
      \begin{minipage}[b]{0.48\textwidth}
        \centering
        \includegraphics[width=\textwidth]{../image/3zonerefine.png}
        \caption{Refined Zones}
      \end{minipage}\hfill
      \begin{minipage}[b]{0.48\textwidth}
        \centering
        \includegraphics[width=\textwidth]{../image/3zones_not_refined.png}
        \caption{Not Refined Zones}
      \end{minipage}
    \end{figure}
\end{frame}

\begin{frame}{Comparison of Refined and Not Refined Zones}
    \begin{figure}
      \centering
      \begin{minipage}[b]{0.40\textwidth}
        \centering
        \includegraphics[width=\textwidth]{../image/ACJAsmin_refine.png}
        \caption{ACJAsmin Refined}
      \end{minipage}\hfill
      \begin{minipage}[b]{0.40\textwidth}
        \centering
        \includegraphics[width=\textwidth]{../image/Jasmin_hole.png}
        \caption{Jasmin Hole}
      \end{minipage}
    
    \end{figure}
\end{frame}

\begin{frame}
\begin{figure}
\begin{minipage}[b]{0.40\textwidth}
    \centering
    \includegraphics[width=\textwidth]{../image/ACJasmin_not_refined.png}
    \caption{ACJAsmin Not Refined}
  \end{minipage}\hfill
  \begin{minipage}[b]{0.40\textwidth}
    \centering
    \includegraphics[width=\textwidth]{../image/Jasmin_not_refined_roof.png}
    \caption{Jasmin Not Refined Roof}
  \end{minipage}
\end{figure}

\end{frame}


\begin{frame}{Comparison of ACJASMIN Results}
    \begin{figure}
      \centering
      \begin{minipage}[b]{0.48\textwidth}
        \centering
        \includegraphics[width=\textwidth]{../image/ACJASMIN2000_0.04_refine.png}
        \caption{ACJASMIN Refined}
      \end{minipage}\hfill
      \begin{minipage}[b]{0.48\textwidth}
        \centering
        \includegraphics[width=\textwidth]{../image/ACJASMIN_2000_0.04.png}
        \caption{ACJASMIN Not Refined}
      \end{minipage}
    \end{figure}
    \end{frame}


\subsection{Performance}
\begin{frame}{Results from the Start of Internship and After Grid Simplify}

    \begin{figure}
      \centering
      
      % Première image : Avant simplification
      \begin{minipage}[b]{0.48\textwidth}
        \centering
        \includegraphics[width=\textwidth]{../image/tzones.png}
        \caption{Result from the start of internship. Shape detection took 124s and the kinetic space partition 0.2s. When compared to the size and the complexity of the meeting room mesh, it was kind of disappointing.}
      \end{minipage}\hfill
      
      % Deuxième image : Après simplification
      \begin{minipage}[b]{0.48\textwidth}
        \centering
        \includegraphics[width=\textwidth]{../image/3zones_final.png}
        \caption{Result after grid simplify. Execution time: 1.6s for shape detection and 1.1s for space partition.}
      \end{minipage}
    
    \end{figure}
    
    \end{frame}
\begin{frame}{Execution Time with Our Workflow on Three Zones}
    \begin{table}
        \centering
        \begin{tabular}{lcccc}
            Parameters & Default & min.region.size=2000 & min.region.size=500 & min.region.size=100 \\
            Shape detection & 7.97s & 8.18s & 5.76s & 1.62s \\
            Kinetic space partition & 0.22s & 0.22s & 0.36s & 1.35s \\
            Total execution & 8.2s & 8.41s & 6.13s & 2.98s \\
        \end{tabular}
        \caption{Execution time with our workflow on Three Zones}
        \label{tab:execution_time}
    \end{table}
\end{frame}

\section{Conclusion}
\begin{frame}{Conclusion}
    
\end{frame}

\section{Refereces}
\begin{frame}{bib}
    
\end{frame}

\end{document}
